%% Copyright (c) 2002, 2010 Sam Williams
%% Copyright (c) 2010 Richard M. Stallman
%% Permission is granted to copy, distribute and/or modify this
%% document under the terms of the GNU Free Documentation License,
%% Version 1.3 or any later version published by the Free Software
%% Foundation; with no Invariant Sections, no Front-Cover Texts, and
%% no Back-Cover Texts. A copy of the license is included in the
%% file called ``gfdl.tex''.


\chapter{Открытый код} \label{chapter:open source}

[РМС: В этой главе я позволил себе удалить кое-что -- текст, который рассказывал об открытом коде и не имел отношения к моей жизни или работе.]

В ноябре 1995 года Питер Салус, член фонда свободного ПО и автор книги \textit{A Quarter Century of Unix}, обращается к участникам \enquote{системных обсуждений} почтовой рассылки проекта GNU. Он приглашает их в Кембридж, на конференцию по свободно распространяемому программному обеспечению. Конференция эта планируется на февраль 1996 года, спонсируется фондом свободного ПО, и председательствовать на ней будет сам Питер Салус. Она должна стать первой конференцией, полностью посвящённой свободным программам. На конференцию приглашают разработчиков и пользователей всех крупных свободных проектов: GNU, Linux, FreeBSD, NetBSD, 386BSD, Perl, Tcl/Tk -- всех программ, чей код свободно доступен для чтения, редактирования и раздачи. Вот что пишет Салус:

\begin{quote}
За последние 15 лет свободное и бесплатное ПО стало повсеместным. Эта конференция соберёт разработчиков и дистрибьюторов свободных программ. Планируются консультации, учебные лекции, доклады -- в частности, от Линуса Торвальдса и Ричарда Столлмана.\footnote{Peter Salus, \enquote{FYI-Conference on Freely Redistributable Software, 2/2, Cambridge} (1995), \url{http://bat8.inria.fr/~lang/hotlist/free/licence/fsf96/call-for-papers.html}.}
\end{quote}

Один из тех, кто получил письмо Салуса -- Эрик Реймонд. Он не возглавляет какой-либо проект или компанию, как другие адресаты, но у него неплохая репутация среди хакеров. Главным образом, благодаря участию в свободных проектах и составлению \textit{The New Hacker's Dictionary} (\enquote{Нового словаря хакеров})  -- значительно расширенной версии \textit{The Hacker's Dictionary} (\enquote{Словаря хакеров}) за авторством Гая Стила.

Реймонд очень ждёт конференцию, хоть и не полностью согласен с идеями движения за свободное ПО. Ранее он участвовал в разработке некоторых программ GNU, в частности, GNU Emacs. Участие сошло на нет в 1992 году, после того как Реймонд попросил внести его правки кода в официальную версию GNU Emacs, не обсуждая их со Столлманом. Столлман отверг эту просьбу, и Реймонд обвинил Столлмана в \enquote{микроменеджменте}. \enquote{Ричард поднял шум из-за моих несанкционированных чисток Lisp-библиотек Emacs, -- рассказывает Реймонд, -- меня это так расстроило, что я решил никогда больше с ним не работать}.

Несмотря на это, Реймонд активно участвует в жизни сообщества свободного ПО. Настолько активно, что горячо поддерживает идею Салуса провести конференцию, где наравне выступили бы Ричард Столлман и Линус Торвальдс. Столлман представляет старое, умудрённое опытом поколение хакеров, вскормленных ITS и Unix, тогда как Торвальдс -- олицетворение новой волны Линукс-хакеров. Их совместное выступление покажет единство сообщества, что вдохновит многих, особенно из молодёжи -- например, хакеров вроде Реймонда. \enquote{Я как будто синтез обоих поколений}, -- говорит Реймонд.

Необходимость в такой конференции уже давно назрела из-за ощутимого напряжения между двумя поколениями хакеров. Однако обе группы сходятся кое в чём: им всем хочется узреть финского \textit{вундеркинда} во плоти. К их удивлению, Торвальдс окажется приветливым и очаровательным оратором с живым, самокритичным остроумием и лёгким шведским акцентом. \footnote{Торвальдс родился и жил в Финляндии, но его родной язык -- шведский. \enquote{The Rampantly Unofficial Linus FAQ} at \url{http://catb.org/~esr/faqs/linus/} даёт краткие пояснения на этот счёт:

\begin{quote}
В Финляндии живёт немало шведоязычного населения, около 6\% от общей численности. Они называют себя \textit{finlandssvensk} или \textit{finlandssvenskar} и считают себя финнами; многие из таких семей живут в Финляндии веками. Шведский язык -- один из двух официальных языков Финляндии.
\end{quote}}

Ещё большее удивление, по словам Реймонда, вызывает готовность Торвальдса кидать камни в огород именитых хакеров, включая признанного \enquote{хакера хакеров} Ричарда Столлмана. К концу конференции тонкая дерзость Торвальдса одерживает моральную победу над обоими поколениями хакеров.

\enquote{Это был переломный момент, -- вспоминает Реймонд, -- до 1996 года Столлман был единственным заслуживающим доверия кандидатом на роль лидера хакерской культуры. Да, многие не соглашались с ним, но не публично. Торвальдс же отступился от этого правила}.

На исходе конференции Торвальдс окончательно ломает все негласные табу. Идёт обсуждение растущего рыночного господства Microsoft Windows, и Торвальдс заявляет, что одно время был фанатом программы Microsoft PowerPoint. В глазах хакеров старой школы это всё равно, что хвастаться своими рабами на съезде аболиционистов. Но для Торвальдса и молодого поколения хакеров это просто здравый смысл. Зачем ради принципа избегать удобных собственнических программ? Производители всё равно делают и будут делать такие программы. Рано или поздно наступает момент, когда свобода требует жертв, и человек, для которого свобода -- священная самоцель, идёт на эти жертвы, но другие видят в этом фанатичное самоотречение. Для этих других хакерство было не фанатизмом и самоотречением, а способом сделать дело, причём вполне конкретное дело.

\enquote{Это было довольно шокирующе, -- вспоминает Реймонд, -- но в тот период он быстро набирал влияние, поэтому мог позволить себе такое}.

Столлман, в свою очередь, не ощущает никакого напряжения на конференции -- наверное, потому что пропустил заявление Торвальдса. Но всё же и ему довелось отведать коронной дерзости Линуса. \enquote{В документации Linux был пункт, предписывающий распечатать стандарты программирования GNU и порвать их на куски, -- приводит Столлман пример, -- причём, если приглядеться, становилось понятно, что такую реакцию у него вызвала малозначащая мелочь -- рекомендация по оформлению отступов в коде на языке С}.

\enquote{Хорошо, ты не согласен с какими-то нашими стандартами. Это нормально, но необязательно же заявлять об этом в такой противной манере. Ты бы мог просто сказать: \enquote{Я думаю, отступы в коде нужно оформлять вот так}. И всё. Враждебность ни к чему}.

Впрочем, многие хакеры \enquote{из молодых} позитивно воспринимают заявления и комментарии Торвальдса. Реймонд получает лишнее доказательство: линия раздора в сообществе проходит почти точно по границе поколений. Многие Линукс-хакеры, подобно Торвальдсу, росли уже в окружении несвободных программ. Они начали вносить свой вклад в свободное ПО, не ощущая никакой несправедливости в собственническом софте. Для большинства из них выбор между этими категориями программ -- всего лишь вопрос удобства. Если программа полнофункциональна и технически качественна, то они не видят причин отвергать её из-за одной только лицензии. Пусть однажды хакеры разработают свободную альтернативу PowerPoint, но пока её нет, почему бы не пользоваться несвободным PowerPoint от Microsoft?

Это главная причина растущих разногласий в сообществе. По одну сторону баррикад оказались старые хакеры, для которых самоценна свобода, по другую -- люди, которым важны мощь и надёжность программ. Столлман называет эти стороны политическими партиями внутри сообщества. Первая называет себя \enquote{партией свободы}, вторая не изъявила никакого желания подобрать себе название, поэтому Столлман сам называет её несколько пренебрежительно \enquote{партией большинства} или \enquote{партией успешности}, потому что многие её представители провозглашают приоритетной целью: \enquote{больше пользователей}.

С момента запуска проекта GNU у Столлмана сложилась внушающая страх и благоговение репутация как программиста. И репутация бескомпромиссного упрямца в области архитектуры программ и управления людьми. Отчасти этот образ соответствует действительности, но также он служит и удобным оправданием для тех людей, чьи желания не сходятся со словами и действиями Столлмана. Этот удобный для многих образ укрепляется скандальными слухами и ложными домыслами.

К примеру, незадолго до конференции 1996 года случилось настоящее бегство персонала из фонда свободного ПО. Брайан Юманс, директор фонда, нанятый Салусом перед его уходом в отставку, вспоминает: \enquote{В какой-то момент Питер Салус был единственным сотрудником в офисе}. Причина была в исполнительном директоре. Брит Брэдли рассказывал своим друзьям в 1995 году:

\begin{quote}
[Исполнительный директор фонда, имя которого здесь опущено] на прошлой неделе решила вернуться из отпуска. Мы -- Джена Бин, Майк Дрейн и я -- сошлись во мнении, что с таким начальником работать невозможно. Она сделала немало ошибок до того, как ушла в отпуск. Она нередко угрожала увольнением из-за мелочей, а порой вовсе позволяла себе оскорбления в адрес ВСЕХ сотрудников. Мы просили -- много раз -- не возвращать её в начальники, выражали готовность работать с ней, но только как с коллегой. Все наши просьбы проигнорировали. Мы уходим.
\end{quote}

Тогда исполнительный директор выдвинула президенту фонда -- то есть, Столлману -- ультиматум: дать ей полную свободу действий в офисе, иначе она уйдёт в отставку. Столлман отказался дать ей полный контроль над деятельностью фонда, и нашёл ей замену в лице Питера Салуса.

Реймонд, услышав эту историю со стороны, решает, что вина в бегстве персонала лежит на Столлмане, и укрепляется в своём мнении, что личность Столлмана -- причина многих бед и проблем проекта GNU. Он считает, что проблемы вроде роковой задержки Hurd и раскола Lucid-Emacs связаны с управлением разработкой, а не с самой разработкой.

После конференции проходит немного времени, Реймонд принимается за разработку утилиты для работы с электронной почтой под названием \enquote{fetchmail}. По примеру Торвальдса он выпускает программу с обещанием обновлять код как можно чаще. Когда Реймонда захлёстывают потоки отчётов об ошибках и запросов функциональности, ему кажется, что такая модель разработки породит лишь хаос и не принесёт ничего хорошего. Но со временем он видит, что программа выходит на удивление устойчивой. Реймонд анализирует причины успеха модели Торвальдса и приходит к выводу: интернет в ней используется как \enquote{чашка Петри}, где суровый надзор хакеров играет роль естественного отбора. Модель Торвальдса эволюционна и свободна от централизованного планирования.

Более того, Реймонд считает, что Торвальдс обошёл закон Брукса. Брукс был менеджером проекта IBM OS/360, и в 1975 году выпустил книгу \textit{The Mythical Man-Month}, в которой сказал, что рост числа программистов замедляет разработку программы. Многие хакеры также думают, что большая толпа поваров вряд ли сварит вкусный суп. Но модель Торвальдса, как видит Реймонд, ломает этот закон. Приглашая на кухню всё больше и больше поваров, Торвальдс действительно делает суп \textit{вкуснее}. \footnote{Вообще, сам Брукс не формулировал никакого закона. Законом Брукса называют краткое изложение вот этой цитаты из его книги:

\begin{quote}
Создание программы является системной работой, производной коллективных взаимосвязей. Коммуникационные издержки играют очень большую роль, их негативное влияние быстро перерастает любые выгоды от коллективной работы. Поэтому рост числа разработчиков не сокращает время разработки, а увеличивает его.
\end{quote}

\noindent Fred P. Brooks, \textit{The Mythical Man-Month} (Addison Wesley Publishing, 1995).}

Реймонд переносит свои наблюдения на бумагу и кратко зачитывает их перед группой друзей в пенсильванском округе Честер. Эта речь, названная \enquote{Собор и Базар}, противопоставляет \enquote{базарную} модель разработки Торвальдса традиционному и общепринятому \enquote{соборному} подходу.

Об этой речи тепло отзываются, а на Linux Kongress весной 1997 года в Германии она вызывает настоящий восторг.

\enquote{Когда я закончил говорить, грянули настоящие овации, -- вспоминает Реймонд, -- для меня это было важно по двум причинам. Во-первых, это значило, что моя речь их взволновала. Во-вторых -- что она взволновала их настолько, что даже языковой барьер не сыграл роли}.

После этого Реймонд оформляет свои наблюдения и мысли в полноценное эссе, также назвав его \enquote{Собор и Базар}. Название отражает центральную аналогию Реймонда. Раньше программы создавались подобно \enquote{соборам}, где централизованная вертикальная иерархия создавала впечатляющие заранее спланированные конструкции, призванные выдержать испытание временем. Linux же создаётся другим путём -- на \enquote{огромном шумном базаре} интернета с его полной децентрализацией.

Реймонд связывает соборный подход со Столлманом и проектом GNU, создавая этим очередной контраст между Столлманом и Торвальдсом. Столлман -- каноничный пример строителя собора, то есть, \enquote{волшебника} от программирования, который может исчезнуть на 18 месяцев и вернуться с чем-то вроде GCC. Торвальдс же больше похож на гостеприимного хозяина вечеринки. Он позволяет другим обсуждать архитектуру Linux, и вмешивается только когда требуется рассудить спорщиков и принять решение. Базарная модель разработки Торвальдса отражает его собственную непринуждённость. Своей задачей он видит не контроль над процессом, а поддержку идей.

Итог Реймонд подводит следующий: \enquote{Я думаю, самый умный и значимый хак Линуса это не создание ядра, а создание модели разработки ядра}.\footnote{Эрик Реймонд, \enquote{Собор и Базар} (1997).}

Описание этих двух моделей разработки в эссе получилось очень проницательным, но вот привязка соборной модели к Столлману -- явная клевета. На самом деле, разработчики некоторых программ GNU, включая Hurd, ознакомились с моделью Торвальдса и стали следовать ей ещё до того, как Реймонд привлёк к ней всеобщее внимание. Но эссе Реймонда читают тысячи хакеров, что не добавляет хорошей репутации проекту GNU.

Выводы Реймонда о секрете управленческого успеха Торвальдса привлекают внимание тех членов сообщества свободного ПО, что не считают свободу первоочередной целью. Они стараются привлечь бизнес к разработке и использованию свободных программ, описывая их мощными, надёжными, дешёвыми и продвинутыми -- очень соблазнительными для предпринимателей качествами. Реймонд -- самый известный сторонник и проводник таких идей. Скоро они достигают руководства Netscape, чей собственнический браузер стремительно теряет рынок под напором Microsoft Internet Explorer. Руководители Netscape заинтригованы словами Реймонда и решают изменить стратегию. В январе 1998 года появляется официальное сообщение о планируемом открытии исходного кода Netscape Navigator. Этот шаг -- надежда на то, что хакеры присоединятся к разработке браузера.

Генеральный директор Netscape Джим Барксдейл называет \enquote{Собор и Базар} Реймонда главным фактором, что побудил компанию к смене стратегии, и это делает Реймонда знаменитостью среди хакеров. Он приглашает на встречу нескольких человек, включая основателя VA Research Ларри Августина, издателя Тима О'Рейли, а также президента нанотехнологического института Форсайта Кристину Петерсон. \enquote{Повестка дня сводилась к одному вопросу: как использовать решение Netscape, чтобы другие компании последовали этому примеру?}.

Первый же вопрос касается неправильного истолкования слова \enquote{свободный}. Несмотря на то, что Столлман и другие хакеры неоднократно напоминают людям, что \enquote{свободный} не значит \enquote{бесплатный}, большинство предпринимателей и обычных пользователей упорно желают видеть в свободном нечто бесплатное. Пока хакеры не решат эту проблему, движение за свободное ПО будет топтаться на месте, даже с учётом случаев вроде Netscape.

Петерсон, чей институт активно интересуется свободным ПО, предлагает альтернативную формулировку: \enquote{открытый исходный код} (open source).

Её Петерсон придумала ещё раньше, когда обсуждала решение Netscape со своей подругой, специалистом в сфере PR. Петерсон не запомнила, как сложилась эта формулировка, но запомнила, что подруге она не понравилась. \footnote{Malcolm Maclachlan, \enquote{Profit Motive Splits Open Source Movement,} \textit{TechWeb News} (August 26, 1998), \url{http://www.techweb.com/article/showArticle?articleID=29102344}.}

Участники встречи реагируют на формулировку Петерсон совершенно иначе. \enquote{Я колебалась, предлагая её, -- вспоминает Петерсон, -- я не была членом сообщества, поэтому использовала формулировку как бы между прочим, никак её не выделяя}. К её удивлению, формулировка тут же входит в речь окружающих. К исходу встречи её участники, включая Реймонда, выглядят довольными.

Реймонд, впрочем, пока что не использует публично термин \enquote{открытый код} вместо \enquote{свободное ПО}. Но спустя пару дней после запуска проекта Mozilla О'Рейли собирает конференцию, чтобы обсудить положение дел в свободном программном обеспечении. Он хочет назвать конференцию \enquote{саммитом бесплатных программ} и планирует привлечь внимание СМИ и сообщества к самым разным достойным проектам. \enquote{У всех этих ребят было так много общего, что я удивлялся, как это они до сих пор не знают друг друга, -- вспоминает О'Рейли, -- также очень хотелось донести до мира, какое огромное влияние свободное ПО уже оказало на индустрию и культуру. Люди упускали большую часть традиции свободного ПО}.

Составляя список, О'Рейли принимает решение с далекоидущими политическими последствиями -- он ограничивается разработчиками Западного побережья, такими как Ларри Уолл, создатель sendmail Эрик Оллман, разработчик BIND Пол Викси. С некоторыми исключениями, конечно: проживающий в Пенсильвании Реймонд, который уже был на месте в связи с запуском Mozilla, получает приглашение в числе первых. Как и живущий в Вирджинии создатель языка Python Гвидо ван Россум. \enquote{Фрэнк Уилсон, мой главный редактор и мастер языка Python, пригласил Гвидо без согласования со мной, -- вспоминает О'Рейли, -- я был рад видеть его, но сначала это действительно задумывалось как чисто местная тусовка}.

Отсутствие Столлмана в этом списке извне выглядит как пренебрежение в его адрес. \enquote{Я из-за этого решил не идти на мероприятие}, -- вспоминает Перенс. Реймонд соглашается пойти, но убеждает пригласить Столлмана -- впрочем, безрезультатно. Сообщество охватывают слухи о том, как О'Рейли презрительно отнёсся к Столлману,  причём слухи эти подогреваются известным фактом о вражде между ними. Столлман утверждает, что руководства к свободным программам должны также свободно раздаваться и редактироваться, тогда как О'Рейли считает, что рынок несвободных книг с большой прибыльностью делает свободное ПО популярнее и доступнее, и в конечном счёте идёт сообществу на пользу. Это разногласие привело к публичной взаимной неприязни. Очередное столкновение произошло из-за названия конференции -- Столлман настаивал на словах \enquote{свободное ПО} вместо \enquote{бесплатных программ}. Последний термин обозначает программы, которые раздают бесплатно, но без исходного кода, который можно редактировать.

Сам О'Рейли, вспоминая тот период, не вкладывает никакого негатива в своё решение не приглашать Столлмана. \enquote{Я тогда ни разу не встречался с Ричардом лично, но в переписке по электронной почте он показал себя очень негибким и неконтактным. Я позаботился о том, чтобы традиции GNU были достаточно хорошо представлены, пригласив Джона Гилмора и Майкла Тиманна, которых я знал как очень твёрдых поклонников GPL, но которые при этом были куда гибче в диалоге, особенно при обсуждении плюсов и минусов свободного ПО и его традиций. Видя сейчас, к чему это всё привело, я бы хотел исправить тот промах и пригласить Ричарда, но, повторюсь, это не означало презрения в адрес проекта GNU и лично Столлмана}.

Есть в этом пренебрежение или нет, но термин \enquote{открытый код} однозначно имеет успех на конференции. Участники вовсю делятся опытом и обсуждают идеи о том, как улучшить представления людей о свободном ПО. Например, обратить внимание людей на громадный успех GNU/Linux в области интернет-инфраструктуры, с которым успехи Microsoft Windows не идут ни в какое сравнение. Но понемногу обсуждение возвращается к проблемам вокруг термина \enquote{свободное ПО}. О'Рейли вспоминает забавный комментарий Линуса Торвальдса, который также участвовал в конференции.

\enquote{Линус только недавно переехал в Кремниевую долину, и сказал нам, что только совсем недавно узнал, что в английском языке слово \enquote{free}, оказывается, имеет два значения: \enquote{свободный} и \enquote{бесплатный}}.

Майкл Тиманн, основатель Cygnus, предлагает на замену \enquote{свободному ПО} термин \enquote{исходнософт} (sourceware). \enquote{Это слово никого не воодушевило, -- вспоминает О'Рейли, -- и тогда Эрик произносит слова \enquote{открытый код}\hspace{0.01in}}.

Слова эти нравятся некоторым участникам, но о единодушной смене официальной терминологии говорить не приходится. Когда мероприятие подходит к концу, объявляется голосование за выбор нового термина. Согласно О'Рейли, 9 из 15 участников голосуют за \enquote{открытый код}. Возражения остаются, но несмотря на них, участники договариваются использовать именно такую формулировку в общении с прессой. \enquote{Мы хотели выйти оттуда с посланием солидарности}, -- так описывает это О'Рейли.

Проходит совсем немного времени, и термин \enquote{открытый код} прочно обосновывается в общенародном лексиконе. Вскоре после саммита О'Рейли участвует в пресс-конференции с журналистами \textit{New York Times}, \textit{Wall Street Journal}, и других именитых изданий. Ещё через несколько месяцев на обложке журнала \textit{Forbes} появляется лицо Торвальдса вместе с лицами Столлмана, создателя Perl Ларри Уолла, и лидера проекта Apache Брайана Белендорфа. Бизнес открывает для себя открытый код.

Послание солидарности очень важно для участников конференции вроде Тиманна. Его компания весьма успешна в продаже свободных программ и сервисов, но он не понаслышке знает о трудностях, с которыми сталкиваются программисты и предприниматели.

\enquote{Могу уверенно заявить, что традиционная формулировка нередко сбивала с толку, -- рассказывает Тиманн, -- открытый код позиционируется как полезный и приятный выбор для бизнеса. Свободный софт позиционирует себя как морально праведный. Хорошо это или нет, но мы решили присоединиться к партии открытого кода}.

Реймонд звонит Столлману, чтобы сообщить ему о новой терминологии и спросить его мнения на этот счёт. Он спрашивает Ричарда, принимает ли тот термин \enquote{открытый код}. Столлман говорит, что ему надо поразмыслить об этом, но уже понятно, что он откажется. \enquote{Я знал это, потому что уже говорил с ним лично на эту тему}.

Так и происходит. На следующий день Столлман приходит к выводу, что ценности Реймонда и О'Рейли, несомненно, будут доминировать в будущем дискурсе касательно \enquote{открытого кода}, так что лучшим способом сохранить идеи движения за свободное ПО -- придерживаться традиционной терминологии.

К 1998 году Столлман окончательно определяется с позицией относительно \enquote{открытого кода}. Да, агитация с использованием этого термина помогает доносить до людей технические преимущества свободного ПО, а также в мягкой и ненавязчивой форме продвигать идеи свободы. Да, этот термин избавляет от путаницы в понимании, когда свободное считают бесплатным. Но точно так же \enquote{открытый код} отрезает людей от понимания, что софт может уважать их свободу. Поэтому термин этот бесполезен для движения за свободное ПО. По сути, Реймонд и О'Рейли дали своим термином название неидеалистической партии внутри сообщества, с целями которой Столлман не согласен.

Он считает, что их цели связаны с тем, чтобы понравиться бизнесу. Но такое стремление может привести к очень вредным компромиссам, хотя сама по себе поддержка бизнеса это неплохо. \enquote{Правила ведения переговоров учат, что если вы отчаянно хотите получить чьё-то согласие, то вы уже проиграли, -- говорит Столлман, -- чтобы выиграть, вы должны быть полностью готовы сказать \enquote{нет}}. В результате Столлман на конвенте  LinuxWorld 1999 года и Expo, провозглашённом Торвальдсом \enquote{вечером признания сообщества Linux}, умоляет своих собратьев-хакеров не идти на лёгкие компромиссы.

\enquote{Мы показали, как много можем сделать, поэтому нам не нужно отчаянно добиваться сотрудничества с компаниями или ставить наши ценности под угрозу, -- говорит Столлман во время круглого стола, -- пусть предлагают они, и тогда мы согласимся. Мы не должны прекращать делать то, что мы делаем, чтобы они нам помогли. Вы можете сделать один шаг, затем второй, третий, и в конце концов вы придёте к цели. Если же вы будете шагать наполовину или вбок, своей цели вы никогда не достигнете}.

Однако ещё до LinuxWorld Столлман демонстрирует свою готовность дать бой сторонникам открытого кода. Это происходит на организованной О'Рейли конференции, посвящённой языку программирования Perl. На этот раз Столлмана пригласили. И во время обсуждения решения IBM использовать свободный веб-сервер Apache в своих продуктах, Столлман берёт микрофон для зала и резко критикует присутствующего тут же Джона Оустерхаута, создателя скриптового языка Tcl. Столлман говорит, что Оустерхаут \enquote{паразитирует} на сообществе свободного ПО для того, чтобы продвигать собственническую версию Tcl в рамках своего стартапа Scriptics. Ранее Оустерхаут сообщил, что свободная версия языка Tcl будет получать минимум улучшений, что выглядит покупкой одобрения сообщества формальным вкладом в свободное ПО. Столлман в резкой форме отвергает такую позицию и осуждает планы Scriptics. \enquote{Я не думаю, что Scriptics нужна, чтобы Tcl продолжал существовать}, -- говорит он под шипение окружающих.\footnote{\textit{Ibid.}}

\enquote{Получилась довольно отвратительная сцена, -- вспоминает Рич Морин, глава Prime Time Freeware, -- Джон создал несколько заслуживающих уважения вещей: Tcl, Tk, Sprite. Его вклад более чем ощутим}. Несмотря на свои симпатии к Столлману и его позиции, Морин сочувствует всем, кого обеспокоили эти спорные резкие слова.

Столлман и не думает извиняться. \enquote{Критика собственнического софта это не отвратительно. Собственнический софт -- вот что отвратительно. В прошлом Оустерхаут действительно вносил вклад, но теперь Scriptics станет компанией, которая пишет почти полностью собственнический код. Отстаивать свободу на этой конференции означало не соглашаться почти со всеми. Выступая из зала, я мог сказать лишь несколько фраз. И потому нужно было сказать их так, чтобы они не забылись -- то есть, в очень резкой форме}.

\enquote{Меня упрекают в том, что я \enquote{устраиваю сцены}, когда я серьёзно критикую чьё-то поведение, и в то же время снисходительно называют Торвальдса \enquote{дерзким}, когда он высказывается куда более неприятно по поводу каких-то мелочей. Это похоже на двойные стандарты}.

Спорная критика Столлмана в адрес Оустерхаута на время приглушает симпатии Брюса Перенса. В 1998 году Эрик Реймонд предлагает основать Open Source Initiative или OSI -- организацию, которая следила бы за чистотой использования термина \enquote{открытый код} и давала бы консультации бизнесу. Для детальной проработки терминологии Реймонд нанимает Перенса.\footnote{Bruce Perens, \enquote{The Open Source Definition,} The Open Source Initiative (1998), \url{http://www.opensource.org/docs/definition.html}.}

Позже Перенс понимает, что OSI выступает в оппозиции Столлману и фонду свободного ПО, и решает уйти из неё. Но также Перенс понимает и причины, по которым многие хакеры хотят дистанцироваться от Столлмана. \enquote{Мне правда нравится Ричард, я восхищаюсь им, -- говорит Перенс, -- и я думаю, что Ричард бы добился большего, будь он более уравновешен. Например, он мог бы отойти от дел на пару месяцев}.

Вся энергичность Столлмана мало что может противопоставить общественному успеху сторонников открытого кода. В августе 1998 года производитель процессоров Intel покупает часть акций компании Red Hat, и \textit{New York Times} описывает Red Hat как продукт идеологии, \enquote{известной как свободное ПО и открытый код}.\footnote{Amy Harmon, \enquote{For Sale: Free Operating System,} \textit{New York Times} (September 28, 1998), \url{http://www.nytimes.com/library/tech/98/09/biztech/articles/28linux.html}.} Спустя полгода Джон Маркофф пишет статью в Apple Computer, объявляя о принятии компанией Apple сервера Apache \enquote{с открытым кодом} -- прямо в заголовке. \footnote{John Markoff, \enquote{Apple Adopts \enquote{Open Source} for its Server Computers,} \textit{New York Times} (March 17, 1999), \url{http://www.nytimes.com/library/tech/99/03/biztech/articles/17apple.html}.}

Бизнес идёт в том же направлении, вовсю употребляя термин \enquote{открытый код}. В августе 1999 года компания Red Hat, которая уже привычно называет свои продукты софтом с \enquote{открытым кодом}, размещает свои акции на Nasdaq. В декабре то же самое делает и VA Linux, бывшая VA Research, и ставит рекорд прибыльности -- цена одной акции растёт с \$30 до \$300, после чего откатывается к \$239. Акционеры, которым посчастливилось войти на минимуме и выждать до самого конца, получают баснословные 698\% прибыли. Эрик Реймонд, как акционер, владеет пакетом на 36 миллионов долларов. Впрочем, такие высокие цены не продержатся долго, они упадут при крахе доткомов.

Сторонники открытого кода придерживаются простой идеи: чтобы продать свободное ПО бизнесу -- сделайте свободное ПО дружественным к бизнесу. Они считают, что с рынком нужно не бороться, как это делает Столлман и его движение за свободное ПО, а использовать его себе на благо. Зачем ставить себя на положение изгоя, если можно стать звездой и наращивать своё влияние?

Этот подход привёл к успеху открытый код, но не идеалы свободного софта. Чтобы добиться успеха, они опустили самую важную часть движения: понимание свободы как этической ценности. Сегодня мы видим, к чему это привело: почти все дистрибутивы GNU/Linux включают в себя собственнические программы, Торвальдс охотно принимает собственнические прошивки в ядро Linux, а компания Geeknet, которая раньше называлась VA Linux, вовсю строит свой бизнес на несвободных программах. На веб-серверах господствует Apache, который сам по себе свободен, но компании нередко используют модифицированную собственническую версию от IBM.

\enquote{В свои далеко не лучшие дни Ричард думает, что Линус Торвальдс и я сговорились, чтобы украсть его революцию, -- говорит Реймонд, -- его отрицание открытого кода и намеренный подогрев идеологического раскола -- на мой взгляд, следствие его странного сочетания идеализма и рефлекса защиты своей территории. Кто-то думает, что это проявление эго Ричарда. Я так не считаю. Всё совсем наоборот -- он так сильно отождествляет самого себя с идеями свободного ПО, что любая угроза этим идеям для него всё равно что угроза лично ему}.

Столлман отвечает на это: \enquote{Реймонд извращает мои взгляды -- я не считаю, что Торвальдс \enquote{сговорился} с кем-то, потому что хитрости и интриги ему несвойственны. В самих этих заявлениях видно дурное поведение Реймонда. Вместо того, чтобы возражать моим взглядам, он строит какие-то психологические интерпретации. Он приписывает мне самые мерзкие мотивы и мысли, после чего \enquote{выгораживает} меня, предлагая менее мерзкие толкования. Он частенько \enquote{выгораживает} меня таким образом}.

По иронии судьбы, успех открытого кода и его сторонников никак не умаляет лидерскую роль Столлмана, но искажает понимание его роли в глазах общества. Движение за свободное ПО не может похвастать таким широким деловым влиянием и таким широким присутствием в информационном пространстве. Большинство пользователей GNU/Linux даже не знают о существовании этого движения, не говоря уж о философских тонкостях. Они знакомы с дискурсом вокруг открытого кода, и даже не подозревают, что у Столлмана несколько иные взгляды. Поэтому Ричард получает письма и сообщения со словами благодарности за создание и поддержку \enquote{открытого кода}, и каждый раз он терпеливо объясняет, что никогда не поддерживал \enquote{открытый код}. Он использует эту маленькую возможность напомнить людям о ценностях свободного ПО.

Некоторые авторы признают термин \enquote{свободное ПО} в рамках термина  \enquote{FLOSS} -- \enquote{Бесплатное/свободное ПО с открытым кодом} (\enquote{Free/Libre and Open Source Software}). Однако они говорят, что существует одно движение \enquote{FLOSS}, и это всё равно что заявлять, будто в США есть некое единое движение либералов и консерваторов. Более того, взгляды этого единого движения \enquote{FLOSS} в их изложении в точности повторяют взгляды сторонников открытого кода.

Несмотря на всё это, движение за свободное ПО время от времени даёт о себе знать, понемногу накапливая своё значение в абсолютном выражении. Оно и не думает сложить оружие, оно упорно продолжает отделять свои ценности от идей сторонников открытого кода. \enquote{Одна из главных черт Столлмана в том, что он не двигается с места, -- говорит Ян Мёрдок, -- он готов ждать десятилетиями, пока люди не придут к нему сами, осознав такую необходимость}.

Мёрдок считает, что такая несокрушимость дорогого стоит, во всех смыслах. Пусть Столлман больше не единственный лидер сообщества свободного ПО, но он остаётся единственной путеводной звездой к настоящей, бескомпромиссной свободе. \enquote{Вы всегда можете быть уверены, что он не отступится от своих взглядов, -- говорит Мёрдок, -- большинство людей на это неспособны. Вы можете соглашаться со Столлманом или нет, но уважение он заслужил безоговорочно}.
