%% Copyright (c) 2002, 2010 Sam Williams
%% Copyright (c) 2010 Richard M. Stallman
%% Permission is granted to copy, distribute and/or modify this
%% document under the terms of the GNU Free Documentation License,
%% Version 1.3 or any later version published by the Free Software
%% Foundation; with no Invariant Sections, no Front-Cover Texts, and
%% no Back-Cover Texts. A copy of the license is included in the
%% file called ``gfdl.tex''.

\chapter[Предисловие от Ричарда Столлмана]{Предисловие\\от Ричарда Столлмана}

Цель этого издания -- объединить силу моих знаний с живым взглядом журналиста Вильямса. Пусть читатель решит, насколько хорошо это получилось.

Впервые текст английского издания этой книги я прочитал в 2009 году, когда меня попросили помочь с переводом книги на французский. Это повлекло за собой не только мелкие правки.

Нужно было поправить многие отсылки к фактам. К тому же, Вильямс -- не программист, и потому несколько затуманил важные технические и юридические тонкости, вроде разницы между редактированием кода уже созданной программы, и воплощением некоторых её идей в совершенно новой программе. Так, в первом издании говорилось, что программы Gosmacs и GNU Emacs -- результат редактирования кода оригинального Emacs для PDP-10, что совсем не так. Или ещё хуже: Linux ошибочно назвали \enquote{версией Minix}. Позже компания SCO повторила эту нелепость в своём иске против IBM и Линуса Торвальдса, и её опровергал уже сам Эндрю Таненбаум, создатель Minix.

Также первое издание нагнетает слишком много эмоций вокруг некоторых событий. К примеру, там говорится, что я \enquote{категорически избегал} Линукса в 1992 году, а в 1993 я \enquote{резко передумал} и решил спонсировать Debian GNU/Linux. Тогда как мой интерес к Линуксу в 1993 году и его отсутствие в 1992 -- не более чем прагматизм, имеющий одну цель: довести систему GNU до готовности. Начало работ над ядром GNU Hurd в 1990 году -- такой же практичный шаг на пути к означенной цели.

В общем, оригинальное издание во многих местах получилось ошибочным и бестолковым. Всё это нужно было исправить, но сделать это было очень трудно, не нарушив цельности повествования, и не переписав вообще всё. Было предложение использовать для поправок сноски или примечания, но в большинстве глав таких сносок оказалось бы непомерно много. К тому же, некоторые ошибки стали слишком общепринятыми, чтобы их можно было исправить какими-то сносками. Внутритекстовые отступления сделали бы повествование рыхлым и трудночитаемым, а отдельные примечания читатель быстро начал бы пропускать, утомившись прыгать туда-сюда по книге. Так что я внёс исправления прямо в текст.

Тем не менее, факты и высказывания, выходящие за пределы моих познаний, я проверять не стал. Тут я полностью положился на авторитет Вильямса.

Редакция Вильямса содержала много высказываний с критикой в мой адрес. Все эти высказывания я оставил, только добавил возражения на них, где это было к месту. Я почти ничего не удалял, если не считать \autoref{главы об открытом коде}, где я вычеркнул кое-что, не имеющее отношения к моей жизни или работе. Также я сохранил и местами прокомментировал личные суждения Вильямса с критикой в мой адрес, если только они не искажали факты или сведения о технологиях. А вот его утверждения касательно моей работы, моих мыслей и чувств я исправлял очень вольно. Там, где Вильямс делится своими впечатлениями, я ничего не менял. Все мои правки помечены буквами \enquote{РМС}.

В этом издании полноценная система, совмещающая GNU и Linux, всегда именуется \enquote{GNU/Linux}, а просто \enquote{Linux} всегда обозначает только ядро, созданное Торвальдсом. Исключение -- цитаты, где отступление от этого правила явно обозначено пометкой \enquote{[\textit{sic}]}. Загляните на сайт проекта GNU \url{http://www.gnu.org/gnu/gnu-linux-faq.html}, если хотите узнать, почему полноценную, завершённую систему неправильно называть просто \enquote{Linux}.

Хочу поблагодарить Джона Салливана за массу полезной критики и предложений.
